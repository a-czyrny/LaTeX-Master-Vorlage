%Dokumenteinstellungen und Anpassungen
%Dokumentenklasse "scrbook" - Erweitert um den Verweis auf die Verzeichnisse und Texteigenschaften
\documentclass[chapterprefix=true, 12pt, a4paper, oneside, parskip=half, listof=totoc, bibliography=totoc, numbers=noendperiod]{scrbook}


%Anpassung der Seitenränder (Standard bottom ca. 52mm anbzüglich von ca. 4mm für die nach oben rechts gewanderte Seitenzahl)
\usepackage[bottom=48mm,left=25mm,right=25mm]{geometry}

%Tweaks für scrbook
\usepackage{scrhack}

%Blindtext
\usepackage{blindtext}

%Erlaubt unteranderem Umbrücke captions
\usepackage{caption}

%Stichwortverzeichnis
\usepackage{imakeidx}

%Kompakte Listen
\usepackage{paralist}

%Zitate besser formatieren und darstellen
\usepackage{epigraph}

%Glossar, Stichworverzeichnis (Akronyme werden als eigene Liste aufgeführt)
\usepackage[toc, acronym]{glossaries} 

%Anpassung von Kopf- und Fußzeile
%beinflusst die erste Seite des Kapitels
\usepackage[automark,headsepline]{scrlayer-scrpage}
\automark{chapter}
\ihead{\leftmark}
\chead{}
\ohead{\thepage}
\ifoot*{}
\cfoot[\thepage]{}
\cfoot*{}
\ofoot*{\thepage}
\pagestyle{scrheadings}

%Auskommentieren für die Verkleinerung des vertikalen Abstandes eines neuen Kapitels
%\renewcommand*{\chapterheadstartvskip}{\vspace*{.25\baselineskip}}

%Zeilenabstand 1,5
\usepackage[onehalfspacing]{setspace}

%Verbesserte Darstellung der Buchstaben zueinander
\usepackage[stretch=10]{microtype}

%Deutsche Bezeichnungen für angezeigte Namen (z.B. Innhaltsverzeichnis etc.)
\usepackage[ngerman]{babel}

%Unterstützung von Umlauten und anderen Sonderzeichen (UTF-8)
\usepackage{lmodern}
\usepackage[utf8]{luainputenc}
\usepackage[T1]{fontenc}

%Einfachere Zitate
\usepackage{epigraph}

%Unterstützung der H positionierung (keine automatische Verschiebung eingefügter Elemente)
\usepackage{float} 

%Erlaubt Umbrüche innerhalb von Tabellen
\usepackage{tabularx}

%Erlaubt Seitenumbrüche innerhalb von Tabellen
\usepackage{longtable}

%Erlaubt die Darstellung von Sourcecode mit Highlighting
\usepackage{listings}

%Definierung eigener Farben bei nutzung eines selbst vergebene Namens
\usepackage[table,xcdraw]{xcolor}

%Vektorgrafiken
\usepackage{tikz}

%Grafiken (wie jpg, png, etc.)
\usepackage{graphicx}

%Grafiken von Text umlaufen lassen
\usepackage{wrapfig}

%Ermöglicht Verknüpfungen innerhalb des Dokumentes (e.g. for PDF), Links werden durch "hidelink" nicht explizit hervorgehoben
\usepackage[hidelinks,german]{hyperref}

%Einbindung und Verwaltung von Literaturverzeichnissen
\usepackage{csquotes} %wird von biber benötigt
\usepackage[style=alphabetic, backend=biber, bibencoding=ascii]{biblatex}
\addbibresource{references/references.bib}
%Anpassung der Überschriften
\addtokomafont{disposition}{\rmfamily}

%Zusätzliche Farben
\definecolor{darkgreen}{RGB}{0,100,0}

%Umbenennungen
\renewcommand{\lstlistlistingname}{Quelltextverzeichnis}

%Pluszeichen in der Referenc beim zitieren ausblenden
\renewcommand*{\labelalphaothers}{}

%Anpassugen zur Quelltextdarstellung, kann bei Bedarf überschrieben werden (z.B. wenn unterschiedliche Sprachen zum Einsatz kommen)
\renewcommand{\lstlistingname}{Codeauszug}
\lstset{
	language=Java,
	numbers=left,
	columns=fullflexible,
	aboveskip=5pt,
	belowskip=10pt,
	basicstyle=\small\ttfamily,
	backgroundcolor=\color{black!5},
	commentstyle=\color{darkgreen},
	keywordstyle=\color{blue},
	stringstyle=\color{gray},
	showspaces=false,
	showstringspaces=false,
	showtabs=false,
	xleftmargin=16pt,
	xrightmargin=0pt,
	framesep=5pt,
	framerule=3pt,
	frame=leftline,
	rulecolor=\color{green},
	tabsize=2,
	breaklines=true,
	breakatwhitespace=true,
	prebreak={\mbox{$\hookleftarrow$}}
}

%Anpassungen für das Abkürzungsverzeichnis
\newglossarystyle{dottedlocations}{%
	\renewcommand*{\glossaryentryfield}[5]{%
		\item[\glsentryitem{##1}\glstarget{##1}{##2}] \emph{##3}%
		\unskip\leaders\hbox to 2.9mm{\hss.}\hfill##5}%
	\renewcommand*{\glsgroupskip}{}%
}

%Titelformen - gewünschtes Layout einkommentieren

%%Graduation
\makeatletter

\newcommand*{\gradeType}[1]{\gdef\@gradeType{#1}}
\newcommand*{\firstExaminer}[1]{\gdef\@firstExaminer{#1}}
\newcommand*{\secondExaminer}[1]{\gdef\@secondExaminer{#1}}
\newcommand*{\matrikelnr}[1]{\gdef\@matrikelnr{#1}}
\newcommand*{\submitDate}[1]{\gdef\@submitDate{#1}}

\renewcommand*{\maketitle}{
	\begin{titlepage}
		\newgeometry{left=2.5cm,right=2.5cm,top=9.0cm,bottom=2.5cm}
		\begin{center}
			\vfill
			{\Large \@title\par}
			\vskip 0.5cm
			{\large \bfseries Abschlussarbeit\par}
			\vskip 0.5cm
			{\large zur Erlangung des akademischen Grades\\ \bfseries \@gradeType}
			\vskip 0.5cm
			{\large an der}
			\vskip 0.5cm
			{\large Hochschule für Technik und Wirtschaft Berlin\\ Fachbereich Wirtschaftswissenschaften II\\ Studiengang Angewandte Informatik}
			\vfill
			\begin{flushleft}
				\begin{tabular}[t]{rl}
					1. Prüfer: &\@firstExaminer\\
					2. Prüfer: & \@secondExaminer\\
					\\
					Eingereicht von: &\@author\\
					Matrikelnummer: & \@matrikelnr\\
					Datum der Abgabe: & \@submitDate
				\end{tabular}
			\end{flushleft}
		\end{center}
		\restoregeometry
	\end{titlepage}
}
\makeatother
\gradeType{Master of Science (M.Sc.)}
\secondExaminer{Max Mustermann}

%Research paper
%\include{titles/research_papger}
%\subTitle{Ein optionaler Untertitel der Arbeit}
%\researchPart{A}

%Angaben zur Arbeit und dem Author (von beiden Layouts genutzt)
\title{Dies ist der Titel der Abschlussarbeit der sich auch über mehrere Zeilen erstrecken kann}
\author{Max Mustermann}
\matrikelnr{s0000000}
\submitDate{25.04.2017}
\firstExaminer{Max Mustermann}

%Verzeichnisse generieren
\makeglossaries
\loadglsentries{references/glossary_acronyms.tex}
\setacronymstyle{long-short}

\makeindex[columns=2, title=Stichwortverzeichnis, options= -s resources/styles/indexstyle.ist, intoc]
\indexsetup{level=\chapter*,toclevel=chapter}

%Start des Inhalts
\begin{document}

%Notwendiger Workaround
\pagenumbering{alph}

%Deckblatt erzeugen
\maketitle

\pagenumbering{Roman}

\include{chapter/Vorwort} \clearpage
\include{chapter/Abstract} \clearpage

%Inhaltsverzeichnis
\tableofcontents \newpage

%Hauptteil
\pagenumbering{arabic}
\chapter{Blindtext}
\Blindtext[2][3] 
\blinditemize \clearpage
\chapter{Beispiele} \label{c:beispiele}

Im Kapitel Beispiele (siehe \autoref{c:beispiele}) werden einige wichtige Funktionen und Möglichkeiten von LaTeX demonstriert.

\section{Bild}

Die nachfolgende \autoref{img:beispielbild} demonstriert die Darstellung eines \glqq *.jpg\grqq{} Bildes innerhalb des Textes (beim Einfügen kann auf die Endung verzichtet werden, solange der Name einzigartig ist). Zusätzlich enthält dieses einen Untertitel der über das bereits verwendete Label verlinkt werden kann. Der Untertitel erscheint im Abbildungsverzeichnis.

\begin{figure}[h]
	\centering
	\includegraphics[width=0.7\textwidth]{resources/example}
	\caption{Beispielbild}
	\label{img:beispielbild}
\end{figure}

\section{Quelltext}

Nachfolgend der \autoref{lst:helloworld}.

\begin{lstlisting}[caption={Hello World}, captionpos=b, label={lst:helloworld}]
/**
* The HelloWorldApp class implements an application that
* simply prints "Hello World!" to standard output.
*/
class HelloWorldApp {
	public static void main(String[] args) {
		System.out.println("Hello World!"); // Display the string.
	}
}
\end{lstlisting}

\section{Tabelle}

Nachfolgend \autoref{tbl:DigitalesZertifikat}.

\begin{table}[H]
	\begin{center}
		\renewcommand{\arraystretch}{1.3}
		\begin{tabular}{|l|}
			\hline
			\textbf{Inhaber:}\\
			Alice \\ \hline
			\textbf{Peer (Ersteller):}\\
			Bob \\ \hline
			\textbf{Öffentlicher Schlüssel des Inhabers:}\\
			F2 D2 0E ED FA 4E 9E 0A F2 DD 23 8A 32 44 F3 E9 \\ \hline
			\textbf{Gültigkeit:}\\
			2015-07-01 – 2016-06-30 \\ \hline
		\end{tabular}
	\end{center}
	\caption{Digitales Zertifikat}
	\label{tbl:DigitalesZertifikat}
\end{table}

\section{Literaturverweis}

Weil für die alte und die neue Rechtschreibung verschiedene Trennregeln gelten, sind Deutsch mit alter Rechtschreibung und Deutsch mit neuer Rechtschreibung zwei verschiedene Sprachen (\cite{Knappen2009}, S. 192). \clearpage

%Anhang
\pagenumbering{Alph}

%Abbildungsverzeichnis
\listoffigures \clearpage
%Tabellenverzeichnis
\listoftables \clearpage
%Quelltextverzeichnis
\lstlistoflistings \clearpage
%Stichwortverzeichnis
\printindex \clearpage
%Glossar
\printglossary[title={Glossar}] \clearpage
%Abkürzungsverzeichnis
\printglossary[style=dottedlocations,type=\acronymtype,title={Abkürzungsverzeichnis}] \clearpage

%Literaturverzeichnisse (getrennt nach Stichwort)
\printbibliography[heading=bibintoc, keyword={book}, title={Literaturverzeichnis}]\clearpage
\printbibliography[heading=bibintoc, keyword={online}, title={Onlinequellen}]\clearpage
\printbibliography[heading=bibintoc, keyword={image}, title={Bildquellen}]\clearpage

% Anhang
\appendix

\chapter{}
\addcontentsline{toc}{chapter}{Anhang A}

\section{Diagramm}

\section{Tabelle}

\section{Screenshot}

\section{Graph}

% Eigenständigkeitserklärung
\addchap{Eigenständigkeitserklärung}

Hiermit versichere ich, dass ich die vorliegende Masterarbeit selbstständig und nur unter
Verwendung der angegebenen Quellen und Hilfsmittel verfasst habe. Die Arbeit wurde bisher
in gleicher oder ähnlicher Form keiner anderen Prüfungsbehörde vorgelegt.

\vskip 1cm

Stadt, den xx.xx.xxxx

\vskip 1.5cm

Max Mustermann

\end{document}